\documentclass[12pt, a4paper]{article}
\usepackage[utf8]{inputenc}
\usepackage[croatian]{babel}
\usepackage[margin=1in]{geometry}
\usepackage{amsmath}
\usepackage{amssymb}
\usepackage{amsthm} 
\usepackage{mathtools}
\usepackage{hyperref}
\usepackage{graphicx}
\usepackage{multirow}
\usepackage{fancyhdr} % prored, numeriranje stranice...
\usepackage{amsfonts}
\usepackage{setspace}% za postavljanje praznina
\usepackage{color}
\usepackage{wasysym}
\usepackage{bbm}
\usepackage{lscape}
\usepackage{filecontents}
\usepackage{multicol}
\usepackage{url}
\DeclareMathOperator {\sg}{sg}
\fancyhf{}
\renewcommand{\headrulewidth}{0pt}
\rfoot{\thepage}
\usepackage{paralist}
\usepackage{color}
\usepackage{tikz}
\usetikzlibrary{arrows,positioning,automata}
\PassOptionsToPackage{usenames,dvipsnames,svgnames}{xcolor}
 \newcommand{\HRule}{\rule{\linewidth}{0.5mm}}
\setlength\parindent{24pt}
\DeclarePairedDelimiter\floor{\lfloor}{\rfloor}
\DeclarePairedDelimiter\ceil{\lceil}{\rceil}
\usepackage{listings}
\usepackage{xcolor}
\lstset { %
    language=C++,
    backgroundcolor=\color{black!5}, % set backgroundcolor
    basicstyle=\footnotesize,% basic font setting
}

\begin{document}
\renewcommand{\bibname}{Literatura}%ne želim da mi piše bibliografija već literatura
\begin{titlepage}
\begin{center}
\newtheorem{thm}{Teorem}[section] %u svakom poglavlju teoremi se numeriraju zasebno
\theoremstyle{definition}
\newtheorem{defn}[thm]{Definicija} %enumeracija definicija jednaka enum teorem
\newtheorem{exmp}[thm]{Primjer} % analogno s primjerima
\newtheorem{nap}[thm]{Napomena} % analogno s naomenom

\LARGE Sveučilište u Zagrebu \\[1cm] Prirodoslovno - matematički fakultet \\[1cm] Matematički odsjek\\[3cm]

\Large Eliptičke krivulje u kriptografiji \\[0.5cm]

% Title
\HRule \\[0.4cm]
{ \huge \bfseries 3. domaća zadaća}\\[0.4cm]

\HRule \\[9cm]

% Autor i mentor -> postavljanje ovoga na dno stranice...
\begin{minipage}{0.4\textwidth}
\begin{flushleft} \large
\emph{Student:\\}
Antonio Kovačić
\end{flushleft}
\end{minipage}
\begin{minipage}{0.4\textwidth}
\begin{flushright} \large
\emph{Nastavnik:\\}
Doc. dr. sc. Filip Najman
\end{flushright}
\end{minipage}


\vfill

% Bottom of the page
{\large Zagreb, \today}

\end{center}
\end{titlepage}
\pagestyle{plain} % No headers, just page numbers
\pagenumbering{roman} % Roman numerals
\setcounter{page}{1}
\begingroup
\let\cleardoublepage\relax
\tableofcontents

\newpage


\begin{spacing}{1.5}
\section{Problem}
\begin{enumerate}
\item Nađite racionalan broj $t$ sa svojstvom da za eliptičku krivulju 
\[E \, : \, y^2 =x(x+t)(x+t+38)\]
vrijedi $E(\mathbb{Q})_{\textrm{tors}}=\mathbb{Z}_2 \times \mathbb{Z}_4$.
\item Odredite rang eliptičke krivulje nad $\mathbb{Q}$ zadane jednadžbom
\[y^2=x^3-22x\]
\item Za polinom
\[p(x) = (x - 4)(x - 3)(x - 2)x(x + 1)(x + 2)(x + 3)(x + 4),\]
odredite polinome $q(x), r(x) \in \mathbb{Q}[x]$ takve da vrijedi $p(x)=(q(x))^2-r(x)$ i $\deg r \leq 3$
\end{enumerate}
\newpage
\section{Rješenje}
\subsection{1. zadatak}
Prema \cite[p.~28]{ekk} je opći oblik krivulje s torzijskom podgrupom $\mathbb{Z}_2 \times \mathbb{Z}_4$ oblika:
\begin{equation}
\label{eq:opciOblik}	
y^2=x(x+r^2)(x+s^2), \quad r,s \in \mathbb{Q}
\end{equation}
Prema tome slijedi:
\begin{align}
\label{eq:jdba1}
t &= s^2 \\
t+38 &= r^2
\end{align}
Iz čega slijedi da:
\begin{align}
\label{eq:konZ}
r^2-s^2=38
\end{align}
No ne postoje $(r,s) \in \mathbb{Z}^2$ koji zadovoljavaju \eqref{eq:konZ}. Zaključujemo da je $(r,s) \in (\mathbb{Q}\backslash \mathbb{Z})^2$, a onda i $t \in \mathbb{Q}\backslash\mathbb{Z}$.
Pokušajmo vidjeti postoji li $a \in \mathbb{Z}$ tako da za $t$ vrijedi:
\begin{align}
	\label{eq:odnosZ}
	t=\left(a+\frac{1}{2}\right)^2, \qquad	t+38=\left(a+\frac{3}{2}\right)^2
\end{align}
Iz \eqref{eq:odnosZ} se lako dobije $a=18$, pa je $s=\frac{37}{2}$, a $r=\frac{39}{2}$. Iz čega slijedi da je $\mathbb{Z}_2 \times \mathbb{Z}_4$ podgrupa od  $E(\mathbb{Q})_{\textrm{tors}}$. Još je ostalo provjeriti da $E(\mathbb{Q})_{\textrm{tors}} \neq \mathbb{Z}_2 \times \mathbb{Z}_8$ jer imaju isti oblik, no prema:
\[\textrm{Za } E \, : \, y^2=x(x+r^2)(x+s^2), E(\mathbb{Q})_{\textrm{tors}} \textrm{ izomorfna s } \mathbb{Z}_2 \times \mathbb{Z}_8 \textrm{ akko } \] 
\[rs, rs+r^2, rs+s^2 \textrm{ kvadrati u }\mathbb{Q} \]
No $\frac{37\cdot 39}{4}=\frac{1443}{4}$ nije kvadrat u $\mathbb{Q}$, pa slijedi tražena tvrdnja za $t=\frac{1369}{4}$.

\end{spacing}
\newpage
\nocite{*}
\bibliographystyle{abbrv}
\bibliography{lit}
\end{document}