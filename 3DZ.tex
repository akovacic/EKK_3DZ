\documentclass[12pt, a4paper]{article}
\usepackage[utf8]{inputenc}
\usepackage[croatian]{babel}
\usepackage[margin=1in]{geometry}
\usepackage{amsmath}
\usepackage{amssymb}
\usepackage{amsthm} 
\usepackage{mathtools}
\usepackage{hyperref}
\usepackage{graphicx}
\usepackage{multirow}
\usepackage{fancyhdr} % prored, numeriranje stranice...
\usepackage{amsfonts}
\usepackage{setspace}% za postavljanje praznina
\usepackage{color}
\usepackage{wasysym}
\usepackage{bbm}
\usepackage{lscape}
\usepackage{filecontents}
\usepackage{multicol}
\usepackage{url}
\DeclareMathOperator {\sg}{sg}
\fancyhf{}
\renewcommand{\headrulewidth}{0pt}
\rfoot{\thepage}
\usepackage{paralist}
\usepackage{color}
\usepackage{tikz}
\usetikzlibrary{arrows,positioning,automata}
\PassOptionsToPackage{usenames,dvipsnames,svgnames}{xcolor}
 \newcommand{\HRule}{\rule{\linewidth}{0.5mm}}
\setlength\parindent{24pt}
\DeclarePairedDelimiter\floor{\lfloor}{\rfloor}
\DeclarePairedDelimiter\ceil{\lceil}{\rceil}
\usepackage{listings}
\usepackage{xcolor}
\lstset { %
    language=C++,
    backgroundcolor=\color{black!5}, % set backgroundcolor
    basicstyle=\footnotesize,% basic font setting
}

\begin{document}
\renewcommand{\bibname}{Literatura}%ne želim da mi piše bibliografija već literatura
\begin{titlepage}
\begin{center}
\newtheorem{thm}{Teorem}[section] %u svakom poglavlju teoremi se numeriraju zasebno
\theoremstyle{definition}
\newtheorem{defn}[thm]{Definicija} %enumeracija definicija jednaka enum teorem
\newtheorem{exmp}[thm]{Primjer} % analogno s primjerima
\newtheorem{nap}[thm]{Napomena} % analogno s naomenom

\LARGE Sveučilište u Zagrebu \\[1cm] Prirodoslovno - matematički fakultet \\[1cm] Matematički odsjek\\[3cm]

\Large Eliptičke krivulje u kriptografiji \\[0.5cm]

% Title
\HRule \\[0.4cm]
{ \huge \bfseries 3. domaća zadaća}\\[0.4cm]

\HRule \\[9cm]

% Autor i mentor -> postavljanje ovoga na dno stranice...
\begin{minipage}{0.4\textwidth}
\begin{flushleft} \large
\emph{Student:\\}
Antonio Kovačić
\end{flushleft}
\end{minipage}
\begin{minipage}{0.4\textwidth}
\begin{flushright} \large
\emph{Nastavnik:\\}
Doc. dr. sc. Filip Najman
\end{flushright}
\end{minipage}


\vfill

% Bottom of the page
{\large Zagreb, \today}

\end{center}
\end{titlepage}
\pagestyle{plain} % No headers, just page numbers
\pagenumbering{roman} % Roman numerals
\setcounter{page}{1}
\begingroup
\let\cleardoublepage\relax
\tableofcontents

\newpage


\begin{spacing}{1.5}
\section{Problem}
\begin{enumerate}
\item Nađite racionalan broj $t$ sa svojstvom da za eliptičku krivulju 
\[E \, : \, y^2 =x(x+t)(x+t+38)\]
vrijedi $E(\mathbb{Q})_{\textrm{tors}}=\mathbb{Z}_2 \times \mathbb{Z}_4$.
\item Odredite rang eliptičke krivulje nad $\mathbb{Q}$ zadane jednadžbom
\[y^2=x^3-22x\]
\item Za polinom
\[p(x) = (x - 4)(x - 3)(x - 2)x(x + 1)(x + 2)(x + 3)(x + 4),\]
odredite polinome $q(x), r(x) \in \mathbb{Q}[x]$ takve da vrijedi $p(x)=(q(x))^2-r(x)$ i $\deg r \leq 3$
\end{enumerate}
\newpage
\section{Rješenje}
\subsection{1. zadatak}
Prema \cite[p.~28]{ekk} je opći oblik krivulje s torzijskom podgrupom $\mathbb{Z}_2 \times \mathbb{Z}_4$ oblika:
\begin{equation}
\label{eq:opciOblik}	
y^2=x(x+r^2)(x+s^2), \quad r,s \in \mathbb{Q}
\end{equation}
Prema tome slijedi:
\begin{align}
\label{eq:jdba1}
t &= s^2 \\
t+38 &= r^2
\end{align}
Iz čega slijedi da:
\begin{align}
\label{eq:konZ}
r^2-s^2=38
\end{align}
No ne postoje $(r,s) \in \mathbb{Z}^2$ koji zadovoljavaju \eqref{eq:konZ}. Zaključujemo da je $(r,s) \in (\mathbb{Q}\backslash \mathbb{Z})^2$, a onda i $t \in \mathbb{Q}\backslash\mathbb{Z}$.
Pokušajmo vidjeti postoji li $a \in \mathbb{Z}$ tako da za $t$ vrijedi:
\begin{align}
	\label{eq:odnosZ}
	t=\left(a+\frac{1}{2}\right)^2, \qquad	t+38=\left(a+\frac{3}{2}\right)^2
\end{align}
Iz \eqref{eq:odnosZ} se lako dobije $a=18$, pa je $s=\frac{37}{2}$, a $r=\frac{39}{2}$. Iz čega slijedi da je $\mathbb{Z}_2 \times \mathbb{Z}_4$ podgrupa od  $E(\mathbb{Q})_{\textrm{tors}}$. Još je ostalo provjeriti da $E(\mathbb{Q})_{\textrm{tors}} \neq \mathbb{Z}_2 \times \mathbb{Z}_8$ jer imaju isti oblik, no prema:
\[\textrm{Za } E \, : \, y^2=x(x+r^2)(x+s^2), E(\mathbb{Q})_{\textrm{tors}} \textrm{ izomorfna s } \mathbb{Z}_2 \times \mathbb{Z}_8 \textrm{ akko } \] 
\[rs, rs+r^2, rs+s^2 \textrm{ kvadrati u }\mathbb{Q} \]
No $\frac{37\cdot 39}{4}=\frac{1443}{4}$ nije kvadrat u $\mathbb{Q}$, pa slijedi tražena tvrdnja za $t=\frac{1369}{4}$.
\newpage
\subsection{2. zadatak}
Prema \cite[p.~38]{ekk} za krivulju $E \, : \, y^3=x^3-22x$ je njena pripadna 2-izogena krivulja dana sa $E' \, : \, y^3=x^3 + 4 \cdot 22 x=x^3 +88x$.
Za računanje ranga krivulje $E$ promatramo odgovarajuća preslikavanja $\alpha$ i $\beta$ te određujemo $|Im(\alpha)|$ i $|Im(\beta)|$.\\
Rješenje za $Im(\alpha)$:\\
\[E \, : \, y^3=x^3-22x \to a=0, b=-22\]
Prema tome tražimo $(M,e,N)$  takve da jednadžba:
\[b_1 M^4 + b_2 e^4=N^2\]
ima rješenja, pri čemu vrijedi:
\begin{itemize}
\item $b_1 $ djelitelj od $22$ koji je kvadratno slobodan
\item $b_1 b_2=-22$
\item $(M,e)=1$, gdje $(M,e)$ označava najvećeg zajedničkog djelitelja od $M$ i $e$
\end{itemize}
Naša jednadžba za $b_1$ i $b_2$ izgleda simetrično pa je dovoljno provjeravati egzistenciju rješenja za parove djetitelja:
\begin{enumerate}
\item $b_1 \in \{1,-22\} \to (M,e,N)=(1,0,1)$ za $b_1=1$, a za $b_1=-46$ je rješenje $(M,e,N)=(0,1,1)$
\item $b_1 \in \{-2, 11\} \to$ Jednadžba izgleda:
\[-2M^4 + 11e^4=N^2\]
I ima rješenje $(M,e,N)=(-1,1,3)$
\item $b_1 \in \{2,-11\} \to$ Jednadžba izgleda:
\[-11M^4+2e^4=N^2\] 
U slučaju da je $M$ paran, $e$ mora biti neparan jer $(M,e)=1$, pa onda slijedi i da $N$ mora biti paran:
\[M=2k, N=2l, \textrm{ gdje k i l cijeli brojevi }\]
\[2e^4=11 \cdot 2^4 k^4 + 2^2 l^2 \to e^4=2\cdot (2^2 k^4 + l^3)\]
Pa imamo da $e^4$ mora biti paran, a onda i $e$ mora biti djeljiv s $2$, a to je kontradikcija  jer je i $M$ djeljiv s 2 pa bi imali $2|(M,e)$.\\
U slučaju da je $M$ neparan, slijedi i da $N$ mora biti neparan. 
Imamo onda:
\[M^4 \equiv N^2 \equiv 1 (\textrm{mod} \, 8) \to 11M^4 \equiv 3 (\textrm{mod} \, 8)\]
Pa slijedi da je:
\[2e^4\equiv 4 (\textrm{mod} 8) \to e^4\equiv 2 (\textrm{mod } 4)\]
Provjeravamo vrijedi li gornja kongruencija za $x \in \{0,1,2,3\}$ i lako se dobije da ona nema riješenja.
\item $b \in \{-1, 22\} \to$ Jednadžba izgleda:
\[-1M^4+22e^4=N^2\]
Postupak je analogan kao u prošlom slučaju, odnosno $M$ i $N$ su iste parnosti:
Ako su $M$ i $N$ parni, slijedit će da i $e$ mora biti paran, a to je kontradikcija jer bi bilo da $2|(M,e)$.
Ako pak $M$ i $N$ neparni onda ispada: $22e^4\equiv 2  (\textrm{mod  } 8)$ odnosno $11e^4\equiv 3 e^4\equiv 1  (\textrm{mod } 4)$
za što ispada da nema riješenja uvrštavanjem elemenata iz $\mathbb{Z}_4$.\\
\end{enumerate} 
Zaključujemo da imamo riješenje za 4 različita $b_1$ pa zaključujemo da $|Im(\alpha)|=4$
Analogno riješavamo za $Im(\beta)$:\\
\[b_1'M	^4+b_2'e^4=N^2\].
Gdje $(M,e,N)$ tražimo, a $b_1'$ i $b_2'$ su poznati takvi da vrijedi:
\begin{itemize}
\item $b_1'$ djelitelj od $88$ koji je kvadratno slobodan
\item $b_1' b_2'=88$
\item $(M,e)=1$, gdje $(M,e)$ označava najvećeg zajedničkog djelitelja od $M$ i $e$
\end{itemize}
Iz uvjeta $b_1' b_2'=88$ slijedi da $b_1' > 0$, u suprotnom $b_2'<0$ također pa jednadžba \[b_1'M^4+b_2'e^4=N^2\] uopće nema realnih riješenja. Pozitivni djelitelji od $88$ su: $1,2,4,8,11,22,44,88$ no brojeve $4,8,44$ i $88$ isključujemo kao opcije jer nisu kvadratno slobodnu (dijeli ih 4 (kvadrat broja 2)).
\begin{enumerate}
\item $b_1'=1 \to $ Jednadžba izgleda:
\[M^4+88e^4=N^2\]
Jedno očito riješenje je $(M,e,N)=(1,0,1)$
\item $b_1'=2 \to $ Jednadžba izgleda:
\begin{equation}
\label{eq: bS}
2M^4+4 \cdot 11 e^4=N^2
\end{equation}
Iz jednadžbe vidimo da $2|N$ $\to N=2L \textrm{ L cijeli broj }$:
\[2M^4+4 \cdot 11 e^4=4L^2 \Leftrightarrow M^4 +2 \cdot 11 e^4 =2L^2\]
Sada vidimo da $M^4$ paran, pa jer $M$ cijeli broj slijedi da i $2|M$, pa postoji $K \in \mathbb{Z}$ takav da $M=2K$:
\begin{align}
\label{eq:regis}
2^4 K^4+2 \cdot 11 e^4 &=2 L^2 \to \notag \\
2^3 K^4+11 e^4 &= L^2
\end{align}
Iz zadnje jednadžbe vidimo da $L$ i  $e$ moraju biti iste parnosti:\\
Ukoliko  $e$ i $L$ parni imamo onda $e=2f, L=2J$ pa slijedi:
\[2^3 K+ 2^4 \cdot 11 f =2^2 J \to 2K^4+4\cdot 11=J\]
Zadnji oblik je ekvivalentan jednadžbi \eqref{eq: bS}. Zaključujemo da jednadžba nema netrivijalnih rješenja u $\mathbb{Z}$.
\item $b_1'= 11 \to$ Jednadžba izgleda:
\[11 M^4 + 2^3 e^4=N^2\]
Ima ekvivalentni oblik kao \eqref{eq:regis} pa zaključujemo da nema rješenja.
\item $b_1'=22 \to$ Jednadžba izgleda:
\[22 M^4+ 4e^4=N^2\]
Jedno riješenje je dano sa $(0,1,2)$.
\end{enumerate}
Vidimo da imamo riješenje za $2$ $b_1'$ iz čega slijedi da $|Im(\beta)=2$.\\
Nakon ovako iscrpnog računa, konačno zaključujemo da je:

\[2^r=\frac{|Im(\alpha)||Im(\beta)|}{4}=2 \Rightarrow rank(E(\mathbb{Q}))=1\]
\newpage
\subsection{3. zadatak}
Imamo:
\begin{align}
p(x) &= (x - 4)(x - 3)(x - 2)x(x + 1)(x + 2)(x + 3)(x + 4) \\
\label{eq:polRas}   
     &= x^8+x^7-29x^6 -29 x^5+244 x^4 + 244 x^3 -576 x^2 - 576 x
\end{align}
Znamo da je $\deg r \leq 3$ pa iz toga zaključujemo da su koeficijenti uz $x^4,x^5, x^6, x^7, x^8$ jedanaki za $p(x)$ i za $(q(x))^2$, osim toga zaključujemo da je $\deg q = 4$. Stavimo da je:
\begin{equation}
	\label{eq:polinomQ}
	q(x)=q_4 x^4 + q_3 x^3 + q_2 x^2+ q_1 x+ q_0
\end{equation}
\begin{equation}
\label{eq:polinomR}
	r(x)=r_3 x^3 + r_2 x^2 + r_1 x+r_0	
\end{equation}

Iz toga imamo:
\begin{align}
\label{eq: q2}
(q(x))^2=q_4^2 x^8 + 2q_3 q_4 x^7 +(q_3^2 + 2q_2 q_4)x^6+(2q_2q_3 + 2q_1q_4)x^5+ \notag
\\
(q_2^2+2q_1q_3+2q_0q_4)x^4+(2q_1q_2+2q_0q_3)x^3+(q_1^2+2q_0q_2)x^2+2q_0q_1 x + q_0^2 
\end{align}
\begin{align}
	p(x)=(q(x))^2-r(x)=q_4^2 x^8 + 2q_3 q_4 x^7 +(q_3^2 + 2q_2 q_4)x^6 \notag \\
	    (q_2^2+2q_1q_3+2q_0q_4)x^4+(2q_1q_2+2q_0q_3 - r_3)x^3+ \\ (q_1^2+2q_0q_2 - r_2)x^2+(2q_0q_1 -r_1)x + (q_0^2-r_0)  \notag
\end{align}
Iz toga slijedi da mora vrijediti:
\begin{align}
\label{eq: q4}
q_4^2=1 \to q_4=\pm 1	
\end{align}
Uzimamo $q_4=1$ (da smo uzeli drugačije, dobili bi drugačija rješenja za koeficijente u čijim će se jednadžbama pojavljivati $q_4$, ali to ne bi utjecalo na točnost rješenja, odnosno, rješenja može biti više). Redom ćemo uvrštavati $q_4$ da bi dobili $q_3$, zatim  $q_3$ i $q_4$ za $q_2$,  a onda $q_2, q_3, q_4$ za $q_1$ te na koncu $q_1,q_2, q_3, q_4$ za $q_0$.
\begin{align}
\label{eq: qRj}
2q_3&=1 \to q_3=\frac{1}{2} \notag \\
\frac{1}{4} + 2q_2&=-29 \to 	q_2=\frac{-117}{8} \notag \\
\frac{-117}{8} +2q_1&=-29 \to q_1=\frac{-115}{16} \\
\left(\frac{-117}{8}\right)^2 + \frac{-115}{16}+2q_0 &= 244 \to q_0=\frac{2387}{128} \notag
\end{align}
Iz \eqref{eq: q4} i \eqref{eq: qRj} dolazimo do jednadžbi za koeficijente $r_0, r_1, r_2, r_3$:
\begin{align}
(2q_1q_2+2q_0q_3 - r_3)=244 \to r_3=\frac{-1935}{128} \notag \\
 (q_1^2+2q_0q_2 - r_2)=-576 \to r_2 =\frac{42083}{512} \notag \\
 (2q_0q_1 -r_1)=-576 \to r_2 =\frac{315319}{1024} \\
 (q_0^2-r_0)=0 \to r_0=\frac{5697769}{16384}  \notag
\end{align}
Pa je riješenje dano sa:
\begin{align*}
q(x)=x^4 + \frac{1}{2} x^3-\frac{117}{8}x^2-\frac{115}{16}x + \frac{2387}{128} \\[0.5cm]
r(x)=-\frac{1935}{128} x^3 + \frac{42083}{512} x^2 + \frac{315319}{1024}x + \frac{5697769}{16384}
\end{align*}
\end{spacing}
\newpage
\nocite{*}
\bibliographystyle{abbrv}
\bibliography{lit}
\end{document}